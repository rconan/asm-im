The wind loads applied to FEM of the telescope have been computing with
Computational Fluid Dynamics (CFD) simulations.
The CFD model\cite{GMT.DOC.05211} of the telescope and enclosure is based on CAD drawings released
for the PDR of the GMT mount and the GMT enclosure.
A database of dome seeing, wind loads and heat transfer coefficients from 60 CFD
simulations covering a range of wind speed and orientation,
telescope pointing and enclosure configurations have been
built\cite{GMT.DOC.05214}.

Using the CFD wind loads on the telescope FEM,
the time series of rigid body motions of M1 and M2 segments and of the
truss--to--top-end interface have been computed for 15 CFD cases, as listed in
Table~\ref{tab:cfd-cases}.
The mapping and transformations of the CFD wind loads to FEM wind loads is
described in \cite{GMT.DOC.05506}.

\begin{table}
  \centering
  \begin{tabular}{c}
    -
  \end{tabular}
  \begin{tabular}{ccccc}\toprule
    Zenith & Azimuth & Vents  & Wind screen & Wind speed \\
    deg & deg & - & - & m/s \\\midrule
    30     & 0       & open & stowed & 2 \\ 
    30     & 0       & open & stowed & 7 \\ 
    30     & 0       & closed & deployed & 12 \\ 
    30     & 45       & open & stowed & 2 \\ 
    30     & 45       & open & stowed & 7 \\ 
    30     & 45       & closed & deployed & 12 \\ 
    30     & 45       & closed & deployed & 17 \\ 
    30     & 90       & open & stowed & 2 \\ 
    30     & 90       & open & stowed & 7 \\ 
    30     & 90       & closed & deployed & 12 \\ 
    30     & 135       & open & stowed & 7 \\ 
    30     & 135       & closed & deployed & 12 \\ 
    30     & 180       & open & stowed & 7 \\ 
    30     & 180       & closed & deployed & 12 \\ 
    60     & 45       & open & stowed & 7 \\ \bottomrule
  \end{tabular}
  \label{tab:cfd-cases}
  \caption{List of CFD cases used to compute the wind motions time series.}
\end{table}




%%% Local Variables:
%%% mode: latex
%%% TeX-master: "asm-im"
%%% End:
